%%%%%%%%%%%%%%%%%%%%%%%%%%%%%%%%%%%%%%%%%%%%%%%%%%%
%                                                 %
%                     SECTION                     %
%                                                 %
%%%%%%%%%%%%%%%%%%%%%%%%%%%%%%%%%%%%%%%%%%%%%%%%%%%

\section{Introduction}
\label{sec:sec001}

This document aims to describe the protocol performing a set of tests in the scope of \hyperlink{https://github.com/mida-project/prototype-multi-modality-assistant/releases/tag/v1.0.1-alpha}{v1.0.1-alpha} version from the \hyperlink{https://github.com/mida-project/prototype-multi-modality-assistant}{prototype-multi-modality-assistant} repository of the \hyperlink{https://mida-project.github.io/}{MIDA} project using traditional devices (mouse and keyboard). The goal of the test is to understand the user, performance, efficiency and efficacy metrics in a context of an Artificial Intelligence (AI) diagnosis \cite{fan2018investigating}, hereby denominated as \textbf{Assistant}. The sessions will be recorded via video on the computer and using a record, heat-map and triggered event tools. It is guaranteed the confidentiality of the recordings, which will be used only for academic purposes.

Dividing the activity session into four distinct phases per each three activities representing three different scenarios (Single-Modality vs Multi-Modality vs Assistant). The first three phases took place on an \hyperlink{https://github.com/MIMBCD-UI/testing-guide-breast/tree/master/samples/test_4}{early} stage of the \textit{User Tests} while we were focus to publish on the \hyperlink{https://chi2019.acm.org/}{CHI'19 Conference}. The fourth phase will cover the hereby \textit{User Testing Guide}. Still, we will describe as follows the overall of the four phases to give higher context.

Each scenario will have three patients. In both two scenarios, by supporting our traditional devices, the interaction is made with mouse and keyboard. The first phase, is the \hyperlink{https://docs.google.com/forms/d/1cGmaCGZjeLJhUl_My2wxJ7gcpm7vQRxYhds6Ys0NoSc/edit?usp=sharing}{demographic questionnaire}, where we characterise the Radiologist profile. The second phase is the act of classifying those patients. Radiologists will classify each patient by using the \hyperlink{https://en.wikipedia.org/wiki/BI-RADS}{BIRADS}~\cite{balleyguier2007birads}. On the third phase, we will do several small questionnaires at the end of each scenario using \hyperlink{https://en.wikipedia.org/wiki/NASA-TLX}{NASA-TLX}~\cite{ramkumar2017using}, \hyperlink{https://en.wikipedia.org/wiki/System_usability_scale}{System Usability Scale (SUS)}~\cite{orfanou2015perceived} and measuring the \hyperlink{https://en.wikipedia.org/wiki/BI-RADS}{BIRADS}~\cite{balleyguier2007birads}. Finally, the fourth phase, and most importantly to this \textit{User Testing Guide}, we will provide to the clinicians the \textbf{Assistant} suggestion for the \hyperlink{https://en.wikipedia.org/wiki/BI-RADS}{BIRADS}~\cite{balleyguier2007birads} results and justification of the prognostic. For the user tests we used a three distinct prototype repositories \hyperlink{https://github.com/MIMBCD-UI/prototype-multi-single}{prototype-multi-single}, \hyperlink{https://github.com/MIMBCD-UI/prototype-multi-modality}{prototype-multi-modality} and \hyperlink{https://github.com/mida-project/prototype-multi-modality-assistant}{prototype-multi-modality-assistant}, the three are similar mirrors of the \hyperlink{https://github.com/MIMBCD-UI/prototype-breast-cancer}{prototype-breast-cancer} with minor changes.