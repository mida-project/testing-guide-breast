%%%%%%%%%%%%%%%%%%%%%%%%%%%%%%%%%%%%%%%%%%%%%%%%%%%

%%%%%%%%%%%%%%%%%%%%%%%%%%%%%%%%%%%%%%%%%%%%%%%%%%%
%                                                 %
%                     SECTION                     %
%                                                 %
%%%%%%%%%%%%%%%%%%%%%%%%%%%%%%%%%%%%%%%%%%%%%%%%%%%

\subsection{Post-Task Questionnaire}

Our metrics will refer the \textit{AI-Assisted} involvement against specific goals necessary to satisfy several requirements of our \textit{Assistant}. For our \textbf{Post-Task Questionnaire} we will use \textit{Quantitative Analysis} (QtA) and \textit{Qualitative Analysis} (QlA) in response to our questions (Section \ref{sec:sec003}) to measure the acceptability of our \textit{Assistant} each time a \textit{scenario} is completed (Section \ref{sec:sec003}). From a set of tasks (Section \ref{sec:sec006}) we aim to cover our main scenario, an \textit{AI-Assisted} that we call \textit{Assistant}. Therefore, both QtA and QlA will allow the facilitator to quickly and easily assess the requirements of the given scenario. Our QtA and QlA requirements will have several attributes~\cite{joyce2017healthcare} that make it a good choice for our clinical participants. Those attributes are as follows.

\hfill

%%%%%%%%%%%%%%%%%%%%%%%%%%%%%%%%%%%%%%%%%%%%%%%%%%%

List of the scale attributes:

%%%%%%%%%%%%%%%%%%%%%%%%%%%%%%%%%%%%%%%%%%%%%%%%%%%

\hfill

\begin{itemize}
  \item The requirements are technology agnostic, making it flexible enough;
  \item The requirements are relatively quick and easy to answer;
  \item The requirements provide a single score on a scale that is easily understood;
  \item The requirements are nonproprietary, making it a cost effective tool;
\end{itemize}

\hfill

%%%%%%%%%%%%%%%%%%%%%%%%%%%%%%%%%%%%%%%%%%%%%%%%%%%

\clearpage

The facilitator will explain that the amount of time taken to complete the \textit{tasks} will be measured and that exploratory behaviour outside the \textit{task} flow should not occur until after task completion. At the beginning of each task, the participant will listen the \textit{task} description from the facilitator and begin the task. \textit{Time-on-Task} (ToT) measurements begins when the participant starts the \textit{task}, measured until the end of each \textit{task}.

%%%%%%%%%%%%%%%%%%%%%%%%%%%%%%%%%%%%%%%%%%%%%%%%%%%

%%%%%%%%%%%%%%%%%%%%%%%%%%%%%%%%%%%%%%%%%%%%%%%%%%%
%                                                 %
%                     SECTION                     %
%                                                 %
%%%%%%%%%%%%%%%%%%%%%%%%%%%%%%%%%%%%%%%%%%%%%%%%%%%

\subsection{Training Session}

The participant will receive and overview the test procedure. However, the user will not receive information how to annotate and interact in all degrees of freedom. With the aim of disabling users to get their work done before the test tasks. It will take advantage of a "surprise" acknowledgement.

%%%%%%%%%%%%%%%%%%%%%%%%%%%%%%%%%%%%%%%%%%%%%%%%%%%
%                                                 %
%                     SECTION                     %
%                                                 %
%%%%%%%%%%%%%%%%%%%%%%%%%%%%%%%%%%%%%%%%%%%%%%%%%%%

\subsection{Execution of Tasks}

The \textit{tasks} were derived from test scenarios developed from \textbf{Case Studies}. Due to the range and extent of functionality provided by our \textit{Assistant}, and the short time from which each participant will be available, the \textit{tasks} are the most common and relatively complex of available functions. The \textit{tasks} are the identical for all participants of a given user role in the study.

The \textit{tasks} will be performed by several classes of radiology experience. Professionals from Radiology Seniors, Middles, Juniors and Interns will be performing these \textit{tasks}. On the \textbf{RR} the Radiologist is characterised~\cite{ehrlich2016patient, miglioretti2007radiologist} as a physician who examines and interpret Medical Imaging (MI) \cite{kobashi2017evaluation}, such as X-Rays, CT Scans or MRIs.

%%%%%%%%%%%%%%%%%%%%%%%%%%%%%%%%%%%%%%%%%%%%%%%%%%%

%%%%%%%%%%%%%%%%%%%%%%%%%%%%%%%%%%%%%%%%%%%%%%%%%%%
%                                                 %
%                     SECTION                     %
%                                                 %
%%%%%%%%%%%%%%%%%%%%%%%%%%%%%%%%%%%%%%%%%%%%%%%%%%%

\subsection{Post-Activity Questionnaire}

After completing all \textit{tasks} and scenarios, participants will be asked to complete a questionnaire to classify the \textit{Assistant} according to various parameters regarding the several features. To measure this, we will use an open session \textit{observation} and \textit{interview}~\cite{carayon2015systematic}. We will use this techniques to identify participants' requirements during the various stages of the workflow.

%%%%%%%%%%%%%%%%%%%%%%%%%%%%%%%%%%%%%%%%%%%%%%%%%%%












\section{Measurements}
\label{sec:sec020}

Our measurements refers to user performance measured against specific performance goals necessary to satisfy requirements. \textit{Task} completion success rates, adherence to dialog scripts, error rates and subjective evaluations will be used. \textit{Time-to-Completion} (TtC) of \textit{tasks} will also be collected. The measures are as follows.

\hfill

%%%%%%%%%%%%%%%%%%%%%%%%%%%%%%%%%%%%%%%%%%%%%%%%%%%

The tests are intended to achieve the following measures:

%%%%%%%%%%%%%%%%%%%%%%%%%%%%%%%%%%%%%%%%%%%%%%%%%%%

\hfill

\begin{itemize}
\item BIRADS Classification;
\item Pathology Classification;
\item Time measurement;
\item Number of clicks;
\item Number of errors;
\item Efficiency;
\item Difficulty;
\item Experience;
\end{itemize}

\hfill

%%%%%%%%%%%%%%%%%%%%%%%%%%%%%%%%%%%%%%%%%%%%%%%%%%%

To prioritise recommendations, a method for problem difficulty and degree severity classification, as well as, pathology importance, will be used in the analysis of the collected data during evaluation process. The approach treats problem severity has a combination of several factors. Those factors are measuring the impact of the problem and the frequency of users experiencing issues during the evaluation. Nevertheless, the opinion will also be of chief importance and we will also register the received ones.

\hfill

%%%%%%%%%%%%%%%%%%%%%%%%%%%%%%%%%%%%%%%%%%%%%%%%%%%

Through the questionnaire after the test session, we intend to obtain the answers to the following questions for each \textit{task}:

%%%%%%%%%%%%%%%%%%%%%%%%%%%%%%%%%%%%%%%%%%%%%%%%%%%

\hfill

\begin{itemize}
\item Acceptability of the \textit{Assistant} lesion classification;
\item Acceptability of the \textit{Assistant} interaction;
\item Acceptability of the \textit{Assistant} translation;
\item Acceptability of the \textit{Assistant} available features;
\item \textit{Assistant} degrees of classification;
\item \textit{Assistant} degrees of interaction;
\item \textit{Assistant} degrees of information visualisation;
\end{itemize}

\hfill

%%%%%%%%%%%%%%%%%%%%%%%%%%%%%%%%%%%%%%%%%%%%%%%%%%%