\section{Measurements}
\label{sec:sec007}

Our measurements refers to user performance measured against specific performance goals necessary to satisfy requirements. \textit{Task} completion success rates, adherence to dialog scripts, error rates and subjective evaluations will be used. \textit{Time-to-Completion} (TtC) of \textit{tasks} will also be collected. The measures are as follows.

\hfill

%%%%%%%%%%%%%%%%%%%%%%%%%%%%%%%%%%%%%%%%%%%%%%%%%%%

The tests are intended to achieve the following measures:

%%%%%%%%%%%%%%%%%%%%%%%%%%%%%%%%%%%%%%%%%%%%%%%%%%%

\hfill

\begin{itemize}
\item BIRADS Classification;
\item Pathology Classification;
\item Time measurement;
\item Number of clicks;
\item Number of errors;
\item Efficiency;
\item Difficulty;
\item Experience;
\end{itemize}

\hfill

%%%%%%%%%%%%%%%%%%%%%%%%%%%%%%%%%%%%%%%%%%%%%%%%%%%

To prioritise recommendations, a method for problem difficulty and degree severity classification, as well as, pathology importance, will be used in the analysis of the collected data during evaluation process. The approach treats problem severity has a combination of several factors. Those factors are measuring the impact of the problem and the frequency of users experiencing issues during the evaluation. Nevertheless, the opinion will also be of chief importance and we will also register the received ones.

\hfill

%%%%%%%%%%%%%%%%%%%%%%%%%%%%%%%%%%%%%%%%%%%%%%%%%%%

Through the questionnaire after the test session, we intend to obtain the answers to the following questions for each \textit{task}:

%%%%%%%%%%%%%%%%%%%%%%%%%%%%%%%%%%%%%%%%%%%%%%%%%%%

\hfill

\begin{itemize}
\item Acceptability of the \textbf{Assistant} lesion classification;
\item Acceptability of the \textbf{Assistant} interaction;
\item Acceptability of the \textbf{Assistant} translation;
\item Acceptability of the \textbf{Assistant} available features;
\item \textbf{Assistant} degrees of classification;
\item \textbf{Assistant} degrees of interaction;
\item \textbf{Assistant} degrees of information visualisation;
\end{itemize}

\hfill

%%%%%%%%%%%%%%%%%%%%%%%%%%%%%%%%%%%%%%%%%%%%%%%%%%%