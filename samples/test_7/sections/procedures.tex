\section{Procedures}
\label{sec:sec005}

Participants will take part in the tests at our formed institution protocols (e.g. \hyperlink{http://hff.min-saude.pt/}{Hospital Fernando Fonseca (HFF)}) with the \hyperlink{https://github.com/mida-project/prototype-multi-modality-assistant/releases/tag/v1.0.1-alpha}{v1.0.1-alpha} version of our \hyperlink{https://github.com/mida-project/prototype-multi-modality-assistant}{prototype-multi-modality-assistant} repository. The interaction with the system will be used in a typical \textbf{RR} environment. Note takers and data logger(s) will monitor the sessions for observation in the \textbf{RR}, connected by screen recording feed. The test sessions will be recorded and further analysed.

%%%%%%%%%%%%%%%%%%%%%%%%%%%%%%%%%%%%%%%%%%%%%%%%%%%
%                                                 %
%                     SECTION                     %
%                                                 %
%%%%%%%%%%%%%%%%%%%%%%%%%%%%%%%%%%%%%%%%%%%%%%%%%%%

\subsection{Briefing}

A presentation of the \textbf{Assistant} and it's use and capabilities will be made. Participants will be presented to the available interactions and will be explained how to interact with the prototype, underlining the limitations. The facilitator will brief the participants on the \textbf{Assistant} application and instruct the participant that they are evaluating the application, rather than the facilitator evaluating the participant. Participants will sign an informed consent that acknowledges: the participation is voluntary, that participation can cease at any time, and that the session will be videotaped but their privacy of identification will be granted. The facilitator will ask the participant if they have any question.

%%%%%%%%%%%%%%%%%%%%%%%%%%%%%%%%%%%%%%%%%%%%%%%%%%%

%%%%%%%%%%%%%%%%%%%%%%%%%%%%%%%%%%%%%%%%%%%%%%%%%%%
%                                                 %
%                     SECTION                     %
%                                                 %
%%%%%%%%%%%%%%%%%%%%%%%%%%%%%%%%%%%%%%%%%%%%%%%%%%%

\subsection{Post-Task Questionnaire}

Our metrics will refer the \textit{AI-Assisted} involvement against specific goals necessary to satisfy several requirements of our \textbf{Assistant}. For our \textbf{Post-Task Questionnaire} we will use \textit{Quantitative Analysis} (QtA) and \textit{Qualitative Analysis} (QlA) in response to our questions (Section \ref{sec:sec003}) to measure the acceptability of our \textbf{Assistant} each time a \textit{scenario} is completed (Section \ref{sec:sec003}). From a set of tasks (Section \ref{sec:sec006}) we aim to cover our main scenario, an \textit{AI-Assisted} that we call \textbf{Assistant}. Therefore, both QtA and QlA will allow the facilitator to quickly and easily assess the requirements of the given scenario. Our QtA and QlA requirements will have several attributes~\cite{joyce2017healthcare} that make it a good choice for our clinical participants. Those attributes are as follows.

\hfill

%%%%%%%%%%%%%%%%%%%%%%%%%%%%%%%%%%%%%%%%%%%%%%%%%%%

List of the scale attributes:

%%%%%%%%%%%%%%%%%%%%%%%%%%%%%%%%%%%%%%%%%%%%%%%%%%%

\hfill

\begin{itemize}
  \item The requirements are technology agnostic, making it flexible enough;
  \item The requirements are relatively quick and easy to answer;
  \item The requirements provide a single score on a scale that is easily understood;
  \item The requirements are nonproprietary, making it a cost effective tool;
\end{itemize}

\hfill

%%%%%%%%%%%%%%%%%%%%%%%%%%%%%%%%%%%%%%%%%%%%%%%%%%%

\clearpage

The facilitator will explain that the amount of time taken to complete the \textit{tasks} will be measured and that exploratory behaviour outside the \textit{task} flow should not occur until after task completion. At the beginning of each task, the participant will listen the \textit{task} description from the facilitator and begin the task. \textit{Time-on-Task} (ToT) measurements begins when the participant starts the \textit{task}, measured until the end of each \textit{task}.

%%%%%%%%%%%%%%%%%%%%%%%%%%%%%%%%%%%%%%%%%%%%%%%%%%%

%%%%%%%%%%%%%%%%%%%%%%%%%%%%%%%%%%%%%%%%%%%%%%%%%%%
%                                                 %
%                     SECTION                     %
%                                                 %
%%%%%%%%%%%%%%%%%%%%%%%%%%%%%%%%%%%%%%%%%%%%%%%%%%%

\subsection{Training Session}

The participant will receive and overview the test procedure. However, the user will not receive information how to annotate and interact in all degrees of freedom. With the aim of disabling users to get their work done before the test tasks. It will take advantage of a "surprise" acknowledgement.

%%%%%%%%%%%%%%%%%%%%%%%%%%%%%%%%%%%%%%%%%%%%%%%%%%%
%                                                 %
%                     SECTION                     %
%                                                 %
%%%%%%%%%%%%%%%%%%%%%%%%%%%%%%%%%%%%%%%%%%%%%%%%%%%

\subsection{Execution of Tasks}

The \textit{tasks} were derived from test scenarios developed from \textbf{Case Studies}. Due to the range and extent of functionality provided by our \textbf{Assistant}, and the short time from which each participant will be available, the \textit{tasks} are the most common and relatively complex of available functions. The \textit{tasks} are the identical for all participants of a given user role in the study.

The \textit{tasks} will be performed by several classes of radiology experience. Professionals from Radiology Seniors, Middles, Juniors and Interns will be performing these \textit{tasks}. On the \textbf{RR} the Radiologist is characterised~\cite{ehrlich2016patient, miglioretti2007radiologist} as a physician who examines and interpret Medical Imaging (MI) \cite{kobashi2017evaluation}, such as X-Rays, CT Scans or MRIs.

%%%%%%%%%%%%%%%%%%%%%%%%%%%%%%%%%%%%%%%%%%%%%%%%%%%

%%%%%%%%%%%%%%%%%%%%%%%%%%%%%%%%%%%%%%%%%%%%%%%%%%%
%                                                 %
%                     SECTION                     %
%                                                 %
%%%%%%%%%%%%%%%%%%%%%%%%%%%%%%%%%%%%%%%%%%%%%%%%%%%

\subsection{Post-Activity Questionnaire}

After completing all \textit{tasks} and scenarios, participants will be asked to complete a questionnaire to classify the \textbf{Assistant} according to various parameters regarding the several features. To measure this, we will use an open session \textit{observation} and \textit{interview}~\cite{carayon2015systematic}. We will use this techniques to identify participants' requirements during the various stages of the workflow.

%%%%%%%%%%%%%%%%%%%%%%%%%%%%%%%%%%%%%%%%%%%%%%%%%%%