%%%%%%%%%%%%%%%%%%%%%%%%%%%%%%%%%%%%%%%%%%%%%%%%%%%
%                                                 %
%                     SECTION                     %
%                                                 %
%%%%%%%%%%%%%%%%%%%%%%%%%%%%%%%%%%%%%%%%%%%%%%%%%%%

\section{Methodology}
\label{sec:sec003}

The hereby prototypes are both \hyperlink{}{v0.0.0-alpha} and \hyperlink{}{v0.0.0-alpha} versions of our {\it \hyperlink{}{Prototype A Link}} and {\it \hyperlink{}{Prototype B Link}} prototypes, respectively. The purpose of these prototypes is to involve an \textit{Scope} tool (\textit{Scope Tool Name}) for domain characteristics at a domain level\cite{kobashi2017evaluation}.

%%%%%%%%%%%%%%%%%%%%%%%%%%%%%%%%%%%%%%%%%%%%%%%%%%%
%                                                 %
%                     SECTION                     %
%                                                 %
%%%%%%%%%%%%%%%%%%%%%%%%%%%%%%%%%%%%%%%%%%%%%%%%%%%

\subsection{Environments}

This section describes the user environment over interaction. Supported by this guide, our research aims to conduct an investigation for the several environmental variables and improvements regarding the potentially enhancement that a \textit{Scope} could take in the {\it Environment}.

%%%%%%%%%%%%%%%%%%%%%%%%%%%%%%%%%%%%%%%%%%%%%%%%%%%

\hfill

\begin{figure}[h]
\centering
\includegraphics[width=\textwidth]{sample}
\caption{Figure Title.}
\label{fig:fig001}
\end{figure}

\hfill

%%%%%%%%%%%%%%%%%%%%%%%%%%%%%%%%%%%%%%%%%%%%%%%%%%%

%%%%%%%%%%%%%%%%%%%%%%%%%%%%%%%%%%%%%%%%%%%%%%%%%%%
%                                                 %
%                     SECTION                     %
%                                                 %
%%%%%%%%%%%%%%%%%%%%%%%%%%%%%%%%%%%%%%%%%%%%%%%%%%%

\subsection{Participants}

The participants' responsibilities will be attempting to complete a set of representative task scenarios (Section \ref{sec:sec007}) presented to them in as efficient and timely a manner as possible, and to provide feedback regarding the usability and acceptability of an \textit{Scope}. The participants will be directed to provide honest opinions regarding the user tests of the interacted systems, and to participate in post-session subjective questionnaires and debriefing.

%%%%%%%%%%%%%%%%%%%%%%%%%%%%%%%%%%%%%%%%%%%%%%%%%%%

%%%%%%%%%%%%%%%%%%%%%%%%%%%%%%%%%%%%%%%%%%%%%%%%%%%
%                                                 %
%                     SECTION                     %
%                                                 %
%%%%%%%%%%%%%%%%%%%%%%%%%%%%%%%%%%%%%%%%%%%%%%%%%%%

\subsection{Procedure}

Participants will take part in the tests at our formed institution protocols (\textit{e.g.}, \hyperlink{}{Sample Link}) with both \hyperlink{}{v0.0.0-alpha} and \hyperlink{}{v0.0.0-alpha} versions of our \hyperlink{}{Prototype A Link} and \hyperlink{}{Prototype B Link} prototypes, respectively. The interaction with the system will be used in a typical \textbf{Environment} environment. Note takers and data logger(s) will monitor the sessions for observation in the \textbf{Environment}, connected by screen recording feed. The test sessions will be recorded and further analyzed.

%%%%%%%%%%%%%%%%%%%%%%%%%%%%%%%%%%%%%%%%%%%%%%%%%%%

%%%%%%%%%%%%%%%%%%%%%%%%%%%%%%%%%%%%%%%%%%%%%%%%%%%
%                                                 %
%                     SECTION                     %
%                                                 %
%%%%%%%%%%%%%%%%%%%%%%%%%%%%%%%%%%%%%%%%%%%%%%%%%%%

\subsection{Briefing}

A presentation of the \textit{Scope} and its use and capabilities will be made. Participants will be presented to the available interactions and will be explained how to interact with the prototype, underlining the limitations. The facilitator will brief the participants on the \textit{Scope} system and instruct the participant that they are evaluating the system, rather than the facilitator evaluating the participant. Participants will sign an informed consent that acknowledges: the participation is voluntary, that participation can cease at any time, and that the session will be videotaped but their privacy of identification will be granted. The facilitator will ask the participant if they have any question.

%%%%%%%%%%%%%%%%%%%%%%%%%%%%%%%%%%%%%%%%%%%%%%%%%%%

%%%%%%%%%%%%%%%%%%%%%%%%%%%%%%%%%%%%%%%%%%%%%%%%%%%
%                                                 %
%                     SECTION                     %
%                                                 %
%%%%%%%%%%%%%%%%%%%%%%%%%%%%%%%%%%%%%%%%%%%%%%%%%%%

\subsection{Training}

The participants will receive and overview of the user test procedure, equipment and system. The facilitator will show how to interact with the system and what features are available. We choose this approach, as it provide users the most important concepts to understand and interact with our system. Also, it is of chief importance to give users information of what is and is not available analysis of our \textit{Scope} and what it can do.

%%%%%%%%%%%%%%%%%%%%%%%%%%%%%%%%%%%%%%%%%%%%%%%%%%%

\clearpage