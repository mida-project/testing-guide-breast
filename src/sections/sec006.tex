%%%%%%%%%%%%%%%%%%%%%%%%%%%%%%%%%%%%%%%%%%%%%%%%%%%
%                                                 %
%                     SECTION                     %
%                                                 %
%%%%%%%%%%%%%%%%%%%%%%%%%%%%%%%%%%%%%%%%%%%%%%%%%%%

\section{Evaluation}
\label{sec:sec006}

Introduction of \textit{Scope} agents are significant factors which can naturally affect the performance of a medical workflow. While some prior studies have investigated the functionality of systems, the \textit{Scope} acceptability has mostly been overlooked in the existing Health Informatics (HI) literature regarding a Human-Computer Interaction (HCI) research.

The following Table \ref{table:usability_evaluation_questions} is presenting three main \textit{Research Questions} to have in mind during evaluation. The purpose of this questions is to facilitate systematic user studies regarding our novel \textit{Scope} in an environment and support user stimulation for the introduction of \textit{Scope} methods. The proposed issues, involve various aspects of workflow combined with, either need for satisfaction, nor division of attention.

\hfill

\begin{table}[h]
\centering
\begin{tabular}{l|l}
Number & Research Questions                             	              \\ \hline
RQ1.   & What is the impact of an {\it Scope} system for avoiding     \\
       & different types of errors on environment perception?         \\ \hline
RQ2.   & What are the design techniques for setting appropriate       \\
       & environment expectations of {\it Scope} systems?             \\ \hline
RQ3.   & What is the impact of expectation-setting intervention       \\
       & techniques on {\it Scope} satisfaction and acceptance?       \\ \hline

\end{tabular}
\caption{Research Evaluation Questions}
\label{table:usability_evaluation_questions}
\end{table}

\hfill

The influence of \textit{Scope} is an important variable for our empirical analysis. In fact, we expect that the trust of the user will increase when the user perceived that the \textit{Scope} is giving the right inputs and that there will be a consequent increase of the user trust in our system.

\clearpage

List of associated {\it Research Questions} to respective set of {\it Hypotheses}:

\begin{enumerate}
\item {\bf RQ1.} What is the impact of a {\it Scope} system for avoiding different types of errors on user perception ?
\begin{enumerate}
\item {\bf H1.1.} A {\it Scope} system focused on {\it Some Aspect} will result in higher aspect of accuracy.
\item {\bf H1.2.} A {\it Scope} system focused on {\it Some Other Aspect} will result in higher other aspects?
\end{enumerate}
\item {\bf RQ2.} What are the design techniques for setting appropriate user expectations of {\it Scope} systems ?
\begin{enumerate}
\item {\bf H2.1.} A {\it Scope} system that directly is communicating it accuracy to users will reduce the lack between system accuracy and user perception. 
\item {\bf H2.2.} Providing users explanations (\hyperlink{}{Some Link}) will lead to higher perception of understanding how the {\it Scope} system works.
\item {\bf H2.3.} A first user contact with the system will lead to higher perceived level of control over the {\it Scope} results.
\end{enumerate}
\item {\bf RQ3.} What is the impact of expectation-setting intervention techniques on satisfaction and acceptance of {\it Scope} ?
\begin{enumerate}
\item {\bf H3.1.} In the mediation of an imperfect {\it Scope} system providing users the power of prior interventions will lead to higher acceptance and satisfaction in comparison to a lack of such interventions.
\end{enumerate}
\end{enumerate}