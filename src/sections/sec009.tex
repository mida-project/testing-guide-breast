%%%%%%%%%%%%%%%%%%%%%%%%%%%%%%%%%%%%%%%%%%%%%%%%%%%
%                                                 %
%                     SECTION                     %
%                                                 %
%%%%%%%%%%%%%%%%%%%%%%%%%%%%%%%%%%%%%%%%%%%%%%%%%%%

\section{Goals}
\label{sec:sec009}

The next sections will describe the goals for {\it \hyperlink{}{Prototype A Link}}, {\it \hyperlink{}{Prototype B Link}} and {\it \hyperlink{}{Prototype C Link}} prototype expectations. We will try to assess performance-related metrics such as time and correctness of participants completing \textit{tasks} for our expectations. Our expectations are based of the \hyperlink{}{Results Link} obtained at the lab as \hyperlink{https://www.nngroup.com/articles/pilot-testing/}{pilot tests}.

%%%%%%%%%%%%%%%%%%%%%%%%%%%%%%%%%%%%%%%%%%%%%%%%%%%
%                                                 %
%                     SECTION                     %
%                                                 %
%%%%%%%%%%%%%%%%%%%%%%%%%%%%%%%%%%%%%%%%%%%%%%%%%%%

\subsection{Completion Rate}

\textbf{Completion Rate} is the percentage of test participants who successfully complete the task without critical errors. A critical error is defined as an error that results in an incorrect or incomplete outcome. In other words, the completion rate represents the percentage of participants who, when they are finished with the specified task, have an "output" that is correct.

%%%%%%%%%%%%%%%%%%%%%%%%%%%%%%%%%%%%%%%%%%%%%%%%%%%

\hfill

\textit{A \textbf{Completion Rate} of \textbf{90\%} is the goal for each task in this usability test.}

\hfill

%%%%%%%%%%%%%%%%%%%%%%%%%%%%%%%%%%%%%%%%%%%%%%%%%%%

%%%%%%%%%%%%%%%%%%%%%%%%%%%%%%%%%%%%%%%%%%%%%%%%%%%

\hfill

\textbf{Note:} If a participant requires assistance in order to achieve a correct output then the task will be scored as a critical error and the overall completion rate for the task will be affected.

\hfill

%%%%%%%%%%%%%%%%%%%%%%%%%%%%%%%%%%%%%%%%%%%%%%%%%%%

\clearpage

%%%%%%%%%%%%%%%%%%%%%%%%%%%%%%%%%%%%%%%%%%%%%%%%%%%
%                                                 %
%                     SECTION                     %
%                                                 %
%%%%%%%%%%%%%%%%%%%%%%%%%%%%%%%%%%%%%%%%%%%%%%%%%%%

\subsection{Error-Free Rate}

\textbf{Error-Free Rate} is the percentage of test participants who complete the task without any errors (critical or non-critical errors). A non-critical error is an error that would not have an impact on the final output of the task but would result in the task being completed less efficiently.

%%%%%%%%%%%%%%%%%%%%%%%%%%%%%%%%%%%%%%%%%%%%%%%%%%%

\hfill

\textit{An \textbf{Error-Free Rate} of \textbf{80\%} is the goal for each task in this tests.}

\hfill

%%%%%%%%%%%%%%%%%%%%%%%%%%%%%%%%%%%%%%%%%%%%%%%%%%%

%%%%%%%%%%%%%%%%%%%%%%%%%%%%%%%%%%%%%%%%%%%%%%%%%%%
%                                                 %
%                     SECTION                     %
%                                                 %
%%%%%%%%%%%%%%%%%%%%%%%%%%%%%%%%%%%%%%%%%%%%%%%%%%%

\subsection{Time on Task (ToT)}

The time to complete a scenario is referred to as "Time on Task" (ToT). It is measured from the time the participant begins the scenario to the time which the participant signals completion.

%%%%%%%%%%%%%%%%%%%%%%%%%%%%%%%%%%%%%%%%%%%%%%%%%%%
%                                                 %
%                     SECTION                     %
%                                                 %
%%%%%%%%%%%%%%%%%%%%%%%%%%%%%%%%%%%%%%%%%%%%%%%%%%%

\subsection{Subjective Measures}

Subjective opinions about specific tasks, time to perform each task, features, and functionality will be surveyed. At the end of the test, participants will rate their satisfaction with the overall system. Combined with the interview/debriefing session, these data are used to assess attitudes of the participants.

Measuring subjective outcomes based on participants' experiential goals can pose challenges (Section \ref{sec:sec010}) from which an {\it open-ended} flexible approach is catered to personally meaningful goals. On the other hand advocates of formalized User-Centred Design (UCD) goal exploration condemn such informal interviewing as ineffective and we should take it into consideration.

%%%%%%%%%%%%%%%%%%%%%%%%%%%%%%%%%%%%%%%%%%%%%%%%%%%

%%%%%%%%%%%%%%%%%%%%%%%%%%%%%%%%%%%%%%%%%%%%%%%%%%%
%                                                 %
%                     SECTION                     %
%                                                 %
%%%%%%%%%%%%%%%%%%%%%%%%%%%%%%%%%%%%%%%%%%%%%%%%%%%

\subsection{Case Studies}

The functionality of the prototype will be best demonstrated by a series of case studies. By describing the expected workflow and capabilities of the research study at the \textbf{Environment} specific environment and changes of the workflow by using our system prototype. The study implies the evaluation of domain \textit{Scope} features on several domain aspects. The primary goal of this case studies analysis is to generate a receiver operating characteristic to evaluate the performance and validation of our \textit{Scope}. Let us consider a list of hypothetical use cases for the research investigation that evaluates the interaction and usability performance of the \textit{Scope}. Therefore, the following list will show the preliminary case studies.

\clearpage

List of case studies to analyse our solution prototype:

\hfill

\begin{itemize}
\item Case 1;
\item Case 2;
\item Case 3;
\item Case 4;
\end{itemize}

\hfill